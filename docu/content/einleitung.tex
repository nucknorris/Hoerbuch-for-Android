%!TEX root = ../docu.tex
\section{Einleitung}
\subsection{Motivation}
mobile endgeräte wie tables und smartphones nehmen stetig einen höhere bedeutung im alltag vieler menshcne ein. dies bestätigen aktuelle verzaufszahlen von erwähnten geräten.diese sind in abblidung \ref{sale1}\footnote{http://www.statista.com/statistics/74592/quarterly-worldwide-smartphone-sales-by-operating-system-since-2009/} ersichtlich. sie bestimmten die art und weise der informationsverareitung und beschafft sowie die kommunikation derer die diese geräte benutzen.


\begin{figure}
\begin{center}
\includegraphics[scale=0.6]{images/sale}
\caption{Verkaufszahlen von Smartphones weltweit nach Plattform}
\label{sale1}
\end{center}
\end{figure}


aktueller vorreite der brache ist das auf linux basierende betriebssystem android von google. die wachsende beliebtheit dieses betriebsystems für tablets und smartphone amcht es umso interresanter für entwickler applikationen und services für geräte zu entwickeln die dieses betriebssystem nutzen.

um einen einblick in die entwicklung von applikationen für das mobile betriebsystem haen wir uns daher für die entwicklung einer applikation für die besagte platvor entschieden.

die geplante applikation soll einen vorbestimmten zweck dienen und einen nutzen erfüllen.

\subsection{Vorwort}
Die arbeit soll einblicke in die entwicklung von applikationen für das betriebsystem android gewähren. Es wernden die grundstrukturen von android vorgestellt und erläutert. diese grundstrukcturen sind äußerst wichtig um die art und weise wie applikationen auf der platform ausgeführt werden und vom nutzer verwendet werden.

abshcließend werden verschiedene probleme und lösungswege wärend den konzeptionierung und implementierung sowie der projektbearbeitung erläutert.

dies beinhaltet verschiedene projektorganisatorische elmente wie versionierung und projektplanung (\ref{proj}). die kapitel beschäftigen sich mit vershcienden herrangehensweisen für die projektarbeit in kleinen teams sowie die funktionsweise und anwendung.

desweiteren werden verschiedene schritte von konzeptionierung sowie implementierung verschiedener komponentenstrukturen der applikation diskutiert und ausgewertet. das letzendliche ziel der arbeit ist eine nutzbare zweckgerichtete applikation.

diese appliaktion soll dazu dienen die grundlagen der android plattform zu erlangen. dies beinhaltet die funktionsweise von nativen android appliaktionen (\ref{natand}) sowie der andorid SDK. ein tifes verständis für die eigentliche android betriebssystem erleichter die entwicklung und das verständnis füber verschiedene betriebssystem spezifishcen eigenheiten. deise und weiter grundlagen sollenebenso in dieser arbeit behandelt und deiskutiert werden.
