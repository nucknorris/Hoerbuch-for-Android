%!TEX root = ../docu.tex
\section{Programmstruktur}
\subsection{Allgemeiner Programmaufbau}
\subsubsection{Benutzeroberfläche}
Die Steuerung aller Programmfunktionen wird durch eine Benutzeroberfläche realisiert. Diese Oberfläche wird über die typischen Elemente von Android Smartphones unterstützt. Zum einen sind das die üblichen 3 Systemknöpfe und zum anderen die gegen druck sensible Anzeigefläche\footnote{Touchscreen}.

Die Standrat Eingabeknöpfe bei Android Telefonen sind: \textit{Zurück}, \textit{Home} und \textit{Menü}-Knopf. Diese befinden sich auf allen Android basierenden Smartphones. Einige Knöpfe lassen sich zur Steuerung von Programmfunktionen nutzen andere wiederum nicht.

Der Hauptteil der Steuerung erfolgt über die gegen druck sensible Anzeigefläche des Telefons. Der Aufbau solcher Displays variiert je nach Hersteller und Typ. Die Funktion unterscheidet sich nicht. Das Display erfasst die Druckpunkte von Fingern auf dem Display. Diese Impulse werden in Koordinaten umgerechnet um Interaktionen mit dem Elementen auf dem Display zu ermöglichen. Die errechneten Koordinaten entsprächen jenen die sich unmittelbar unterhalb des Druckpunktes befinden.

Durch diese Interaktion können verschiedene Bedienelemente verwendete werden. Diese reichen von Buttons über Listen die durchlaufen werden können usw.

\subsubsection{Mediaplayer}
Das Abspielen von Audiodateien wird von einen Mediaplayer realisiert. dieser erhält als Eingabeparameter eine Audiodatei in vorm eines Fades einer Datei, welche sich auf dem internen oder externen Speicherbereiches des Smartphones befindet. 

Der Player ist so konzipiert das eine Liste alle Audiodateien im Ordner,indem sich die übergebene Audiodatei befindet erzeugt wird. Diese Liste ist gleichzeitig die Warteschlange des Mediaplayers.

Der Mediaplayer muss nicht eigenständig implementiert werden. Die Android API bietet einen schon implementierten Player für verschiedene gängige Audioformate wie MP3 an. Alle benötigten Decoder-Werkzeuge sind im Player integriert. Für spezielle Audioformate müsste eine passende Erweiterung implementiert oder ein anderer Player verwendet werden.

\subsubsection{Datenbankstrukturen}
\label{Datenbankstrukturen}
Viele moderne Anwendungen benötigen Speicherbereiche um anfallende Anwendungsdaten aus dem Hauptspeicher aus zu lagern. Diese Daten sollen jedoch für einen späteren Zugriff leicht erreichbar bleiben und von andren Anwendungen nicht manipuliert werden können.

Für diesen Zweck gibt es mehrere Möglichkeiten. So ist es zum Beispiel denkbar anfallende Anwendungsdaten zu Verschlüsseln und zusammen mit den Nutzerdaten auf den internen oder externen Telefonspeicher ab zu legen. Diese Daten sind nun durch dritte manipulierbar. Dies beeinträchtigt die Stabilität und Sicherheit der eigenen Anwendung und des gesamten Systems.

Um nun Entwickler die Speicherung von Anwendungsdaten zu erleichtern bietet die Android API eine integrierte Datenbank an. Diese Datenbank bietet nicht nur Schutz der eigenen Daten von anderen Anwendungen, sondern auch den Schutz vom Anwender selbst.

Als Datenbank Struktur wird das quelloffene Datenbank-Management-System SQlite\footnote{http://www.sqlite.org/} verwendet. Es ist sehr ressourcensparend und einfach in die Programmstruktur einzubinden. Während der Laufzeit der Anwendung verbraucht der Datenbankserver nur wenige hundert Kilobyte vom Hauptspeicher.

Alle anfallenden Daten werden in einer einzigen Datei gespeichert. SQlite bietet alle wichtigen Features wie Tabelle, Views, Trigger usw. Es gibt jedoch auch Unterschiede, so ist es zum Beispiel nicht möglich mehre Schreib- und Leseprozesse parallel auszuführen. Alle Datenbankoperation werden daher sequenziell ausgeführt. SQlite biete keine Typsicherheit, fehlerhafte Eingaben werden einfach umgewandelt und gespeichert. Dies sollte bei der Entwicklung von Anwendungen beachtet werden.

Somit hat jede Anwendung welche die SQlite Datenbank implementiert hat, eine eigene separate und vor anderen Anwendungen Geschütze Datenbank.

Daten werden mittels SQL-Syntax abgefragt. Anders als bei vielen Datenbankserver verfügt SQlite keine Benutzer und Zugriffskontrolle. Es gelten die Schrieb- und Leserechte des Dateisystems. Es ist weiterhin nicht möglich Tabellen im vollen Umfang zu verändern, so wie man es von herkömmlichen relationalen Datenbanken gewohnt ist. Der SQL-Befehlt \textit{ALTER TABLE} ermöglicht das Umbenennen von Tabellen und hinzufügen von Spalten.

SQlite ist durch seine einfache Struktur und Handhabung das meistgenutzte Datenbanksystem der Welt.
