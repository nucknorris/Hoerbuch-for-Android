%!TEX root = ../docu.tex
\section{Programmstruktur}

\subsection{Allgemeiner Programmaufbau}

\subsubsection{Benutzeroberfläche}

Die Steuerung aller Programmfunktionen wird durch eine Benutzeroberfläche realisiert. Diese Oberfläche wird über die typischen Elemente von Android Smartphones unterstützt. Zum einen sind das die üblichen Systemknöpfe und zum anderen die gegen den Fingerdruck sensible Anzeigefläche\footnote{engl. Touchscreen}. Die Standard Eingabeknöpfe bei Android-Geräten sind: \textit{Zurück}, \textit{Home} und der \textit{Programmwechsler}-Knopf. Diese befinden sich auf allen aktuellen Android-basierten Smartphones. Einige Knöpfe lassen sich zur Steuerung von Programmfunktionen nutzen. Viele neuen Geräte wie zum Beispiel das Nexus 4\footnote{http://www.google.de/nexus/4/} oder Nexus 7\footnote{http://www.google.de/nexus/7/} haben diese Knöpfe nicht physisch sondern in Form von \emph{Softkeys}, als Darstellung auf dem Display am unteren Bildschirmrand dargestellt. Das hat den Vorteil, dass die Navigationsknöpfe bei Bedarf ausgeblendet werden können und beispielsweise bei der Videowiedergabe die komplette Bildschirmgröße zur Darstellung verwendet werden kann.

Als Haupteingabemethode wird bei modernen Smartphones ein Bildschirm mit Mehrfingergestenerkennung verwendet. Diese berührungsempfindlichen Oberflächen werden zu Steuerung des Gerätes und zur Darstellung von Inhalten auf dem Gerät verwendet. Der Bildschirm erfasst dabei die Berührung des einen oder mehrerer Finger. Die Berührung erzeugt auf dem gleichmäßigen elektrischen Feld des Bildschirms einen geringen Ladungstransport, der von Controllern am Rand erkannt wird. Diese Impulse werden in Koordinaten umgerechnet, wodurch das Gerät genau nachvollziehen kann, wo sich die Finger befinden. Durch diese Interaktion können verschiedene Bedienelemente wie einfache Buttons auch komplexere, mit dem Finger scrollbare, Listen dargestellt werden.

\subsubsection{Mediaplayer}

Das Abspielen von Audiodateien wird von einen Mediaplayer realisiert. Dieser erhält als Eingabeparameter eine Audiodatei. Der Pfad dieser Audiodatei dient zur Identifizierung und wird für das Abspielen benötigt. Der Player ist so konzipiert, dass eine Liste alle Audiodateien im Ordner, indem sich die übergebene Audiodatei befindet, erzeugt wird. Diese Liste ist gleichzeitig die Warteschlange des Mediaplayers. Dieser muss nicht eigenständig implementiert werden. Die Android API bietet einen schon implementierten Player für verschiedene gängige Audioformate wie MP3 an. Alle benötigten Decoder-Werkzeuge sind im Player integriert. Für spezielle Audioformate müsste eine passende Erweiterung implementiert oder ein anderer Player verwendet werden.

\subsubsection{Datenbankstrukturen}
\label{Datenbankstrukturen}

Viele moderne Anwendungen benötigen Speicherbereiche, um anfallende Anwendungsdaten aus dem Hauptspeicher auszulagern. Diese Daten sollen jedoch für einen späteren Zugriff leicht erreichbar bleiben und von anderen Anwendungen nicht manipuliert werden können.

Für diesen Zweck gibt es mehrere Möglichkeiten. So ist es zum Beispiel denkbar anfallende Anwendungsdaten zu verschlüsseln und zusammen mit den Nutzerdaten auf den internen oder externen Telefonspeicher abzulegen. Diese Daten sind nun durch Dritte manipulierbar. Dies beeinträchtigt die Stabilität und Sicherheit der eigenen Anwendung und des gesamten Systems.

Um nun Entwickler die Speicherung von Anwendungsdaten zu erleichtern, ist es möglich durch die Android API eine integrierte Datenbank zu verwenden. Diese bietet nicht nur Schutz der eigenen Daten vor anderen Anwendungen, sondern auch vor dem Anwender selbst.

Als Datenbank wird das quelloffene Datenbank-Management-System SQlite\footnote{http://www.sqlite.org/} verwendet. Es ist ressourcensparend und einfach in die Programmstruktur einzubinden. Während der Laufzeit der Anwendung verbraucht der Datenbankserver nur wenige hundert Kilobyte vom Hauptspeicher.

Alle anfallenden Daten werden in einer einzigen Datei gespeichert. SQlite bietet alle wichtigen Features wie Tabelle, Views, Trigger usw. Es gibt jedoch auch Unterschiede. So ist es zum Beispiel nicht möglich, mehre Schreib- und Leseprozesse parallel auszuführen. Alle Datenbankoperation werden daher sequenziell ausgeführt. SQlite biete keine Typsicherheit. Fehlerhafte Eingaben werden einfach umgewandelt und gespeichert. Dies ist bei der Entwicklung von Anwendungen zu beachten. Somit hat jede Anwendung, welche die SQlite Datenbank implementiert hat, eine eigene separate und vor anderen Anwendungen geschütze Datenbank. Daten werden mittels SQL-Syntax abgefragt. Anders als bei vielen Datenbankserver verfügt SQlite über keine Benutzer und Zugriffskontrolle. Es gelten die Schrieb- und Leserechte des Dateisystems. Es ist weiterhin nicht möglich, Tabellen im vollen Umfang zu verändern, so wie man es von herkömmlichen relationalen Datenbanken gewohnt ist. Der SQL-Befehlt \verb+ALTER TABLE+ ermöglicht das Umbenennen von Tabellen und hinzufügen von Spalten.

SQlite ist durch seine einfache Struktur und Handhabung das meistgenutzte Datenbanksystem der Welt.